\documentclass{ctexart}
\usepackage{graphicx} % Required for inserting images
\usepackage{amsmath}
\usepackage{amssymb}
\usepackage{amsthm}
\usepackage{geometry}
\geometry{a4paper, left=1in, right=1in, top=1in, bottom=1in}
\usepackage{minted}

\title{《人工智能与机器学习基础》第一次实验报告}
\author{PB24000150 李欣宸}
\date{\today}

\begin{document}

\maketitle

\section{实验经过}

\subsection{实验过程}

\begin{enumerate}
    \item 填写代码的空白,补全了 \verb|submission.py| 中线性回归和逻辑回归的代码;
    \item 测试线性回归时发现梯度爆炸,添加了 Weight Clipping 后训练成功,基本达到了解析解的水平(虽然训练步数太长,后续考虑进行规范化);
    \item 考虑到线性回归中 Weight Clipping 属于比较 Ad hoc 的策略,我对输入数据进行规范化处理,并移除了 Weight Clipping 的代码;
    \item 添加了 L1 和 L2 正则化的代码;
    \item 增加了特征数量,构造了对数,倒数特征;
    \item 发现之前写的解析解代码无法处理奇异矩阵,改用 \verb|np.linalg.pinv| 计算伪逆;
    \item 增加了复合特征,即原来的特征两两相乘的结果作为新的特征;
    \item 发现有些标准差为 0 的特征,规范化时会导致除 0 错误,添加了对标准差过小的特征不进行规范化的处理;
    \item 测试了逻辑回归的效果;
    \item 使用 ChatGPT 构造了一些有实际意义的特征,在这些特征的基础上做了一定的删减,得到了能得到良好解析解的一组特征;
    \item 尝试让梯度下降的解尽可能接近解析解;
    \item 书写了 Adam 优化器,并调整了相应的学习率。
\end{enumerate}

\subsection{超参数的调整}

在实验过程中,我主要调整了以下超参数:
\begin{itemize}
    \item 学习率:对于线性回归,使用指数学习率规划器,初始学习率设置为 $0.05$,每次迭代后乘以 $0.9997$;对于逻辑回归,使用固定学习率 $0.002$;
    
    我对线性回归超参数的调节过程是这样的:首先没有学习率规划器,我选择的是能让训练保持稳定、收敛的最大的学习率;之后我发现损失函数在后期会有较大的振荡,因此我添加了学习率规划器,让学习率逐渐减小,从而减小振荡;

    对于逻辑回归,我选择了一个振荡比较小,不是过于明显的学习率;

    \item 训练周期数:由于没有发生明显的过拟合现象,所以训练周期越多,拟合越好,因此线性回归的训练周期数设置为 $50$,逻辑回归的训练周期数设置为 $200$;
    
    \item 正则化系数:由于并没有明显的过拟合现象,因此 L1 和 L2 正则化系数设置为 $0$;
    \item 
    \item Adam 优化器的 $\beta_1,~\beta_2$:采用比较常见的 $\beta_1=0.9,~\beta_2=0.999$;
    
    由于在实验中,我对这两个参数进行调整对结果没有产生任何可观的影响,因此,我最终选择了比较常见的参数。
\end{itemize}

\subsection{正则化处理}

我编写了 L1 和 L2 正则化处理,但是经过测试,即使当特征数量达到 $10^3$ 量级,仍然不会观察到明显的的过拟合现象;使用正则化后,训练集和验证集的训练效果都变差。

这可能是数据集本身规模较大(约 $1.3\times 10^5$ 个训练样本)导致的,因此,在最终的实验中并没有使用正则化处理。

\subsection{其他改进方法}

\subsubsection{规范化}

由于该实验输入特征的量级差异较大(比如 \verb|MWG|,\verb|NWG| 的特征在 $[16, 128]$ 之间,但 \verb|SA|、\verb|SB| 等特征却是 $0$ 或 $1$ 的布尔值),导致梯度下降时学习率太小($10^-6$)收敛太慢,学习率太大($0.01$)容易梯度爆炸。为了解决这个问题,我对输入特征进行了规范化处理。

做出规范化处理后,学习率可以提高到 $0.01$,并且收敛速度也大大提升(可以在 $10$ 周期内收敛);但是,当学习率太大($1$)时,仍然会出现梯度爆炸的现象。

\subsubsection{特征工程}

特征工程是我在该实验中所做的主要改进方法,包括对特征进行变换和构造复合特征。

我想到特征工程的主要原因是,原有的 $14$ 个特征只有 $2$ 的次幂或者是 $0/1$ 的离散值,与运行时间没有简单的线性关系,因此需要对特征采取对数、倒数等变换;且,GPU 上真实的运行时间是各个参数之间互相影响的结果,简单地使用单个特征及其变换无法反映不同参数之间的相互影响,因此可以对特征之间进行乘积等操作,构造新的“复合特征”。

主要的思路是:
\begin{itemize}
    \item 先对原有的 2 次幂的特征进行对数处理;
    \item 再根据数据集的特点,构造一些有用的特征;
    \item 之后再对这些特征构造 2 阶多项式,即构造每个特征的平方和特征间两两相乘的结果作为新的特征;
\end{itemize}

由于 ChatGPT 对 GPU 计算加速有一定的知识,我先使用 ChatGPT 构造了 30 个左右的特征,之后对这些特征进行了删减,删减后除了各个 2 次幂特征的对数及原有特征还有以下的特征:

\begin{itemize}
    \item \verb|wg_threads|:表示工作组中线程的数量。它是 \verb|MDIMC|(本地工作组在 M 维度的大小)和 verb|NDIMC|(本地工作组在 N 维度的大小)的乘积;
    \item \verb|macro_tile_area|、\verb|macro_per_thread|:提取与宏瓦片大小及其在线程中的分布相关的特征;
    \item \verb|per_thread_out|、\verb|per_thread_mn|:反映每线程工作量;
    \item \verb|mn_ratio|、\verb|nm_ratio|:表示宏瓦片在 M 和 N 维度上的长宽比;GPU 上通常正方形的瓦片会更高效,因此这些比例可能有助于理解工作负载的平衡;
\end{itemize}

这些累计 $23$ 个特征已经可以使得 $R^2$ 达到 $0.7$ 以上。

接下来,我们在这些特征的基础上,构造了每个特征的平方和特征间两两相乘的结果作为新的特征,共 $298$ 个,最终解析解的 $R^2$ 能达到 $0.94$;但是,由于尚不明确的原因,梯度下降的结果并不能很好地接近解析解,在使用普通梯度下降优化器时只能达到 $0.85$ 左右,在使用 Adam 优化器时只能达到 $0.91$ 左右。

\subsection{优化器的调整}

由于使用普通梯度下降优化器时,无法使得梯度下降的结果很好地接近解析解,因此我尝试使用 Adam 优化器进行训练。

我想到 Adam 优化器的主要思路是:随机梯度下降优化器对于参数之间相关性较弱,量级相当的情况效果较好,但对于高维,每个维度量级差异较大的情况,往往表现较差。

而对于这个运行时间预测问题,我构造特征的方式属于“广撒网”,特征之间相关性较强,量级差异较大;有些维度与结果关联比较大,而有些维度与结果几乎没有关联,且存在随机噪声。另外很多特征的边缘分布类似长尾分布。所以,这是收敛结果较差的一个很好的理由。

Adam 优化器根据近期梯度的一阶矩、二阶矩信息,自适应地调整每个参数的学习率,比较适合处理高维,每个维度差异较大的情况。采用 Adam 优化器后,$R^2$ 能够达到 $0.91$ 左右,虽然离解析解的 $0.94$ 还有一定的距离。

\section{实验结果}

\subsection{线性回归结果}

下面是使用命令 \mintinline{bash}{python train.py --mode regression}\footnote{学习率、运行步数已经写死在 \texttt{submission.py} 中,通过传参进行调整对运行结果没有影响} 得到的结果:

\begin{verbatim}
    ********** Finish training! **********
    Evaluation results on your eval set: mae: 0.21, R2: 0.91
    ********** The results using analytic solution on eval set **********
    Evaluation results on your eval set: mae: 0.17, R2: 0.94
\end{verbatim}

可以看到,使用梯度下降优化器得到的 $R^2=0.91$,而使用解析解得到的 $R^2=0.94$。虽然它们已经很接近,但还是有明显的差距。这可能与特征之间的关系较为复杂,矩阵较为病态有关。

该模型使用了 $299$ 个参数($298$ 个系数+$1$ 个偏置);由于有同学在更高参数量的模型上可以得到 $R^2=0.97$ 的理论值且不发生过拟合,$R^2$ 达到 $0.94$ 是一个不错的结果,但没有达到上限。

损失函数曲线如图 \ref{fig:regression_loss_curve}:

\begin{figure}[t]
\includegraphics[width=0.48\textwidth]{imgs/train_loss_curve_regression.png}
\includegraphics[width=0.48\textwidth]{imgs/eval_loss_curve_regression.png}
\caption{线性回归训练集和验证集损失函数曲线}
\label{fig:regression_loss_curve}
\end{figure}

由曲线可知,学习率的设置总体合适,且学习率规划器起到了作用,损失函数在后边段的振荡有所减小。

\subsection{逻辑回归结果}

下面是使用命令 \mintinline{bash}{python train.py --mode classification} 得到的结果:

\begin{verbatim}
    ********** Finish training! **********
    Evaluation results on your eval set: F1: 0.94, AUC: 0.99
\end{verbatim}

该模型的结构与线性回归实验相同,损失函数曲线如图 \ref{fig:classification_loss_curve}:

\begin{figure}[t]
    \includegraphics[width=0.48\textwidth]{imgs/train_loss_curve_classification.png}
    \includegraphics[width=0.48\textwidth]{imgs/eval_loss_curve_classification.png}
    \caption{逻辑回归训练集和验证集损失函数曲线}
    \label{fig:classification_loss_curve}
\end{figure}

在曲线中可以看到,该模型损失函数的抖动比线性回归更明显,这是因为逻辑回归的标签只有真假二值,本身就不如线性函数的目标函数平滑。该函数仍然有下降趋势,预示着可能随着训练步数的增加,这个模型的效果还会有较小的提升。

\section{课程反馈}

\subsection{时间消耗}

我在 VSCode 编辑器中安装了 WakaTime 插件,它可以较好地统计编程时间。WakaTime 会排除我打开了 IDE 但实际上在摸鱼的时间。 WakaTime 的统计截图见图 \ref{fig:wakatime_lab1}:

\begin{figure}[t]
    \centering
    \includegraphics[width=0.9\textwidth]{imgs/wakatime_lab1.png}
    \caption{WakaTime 统计的实验时间消耗}
    \label{fig:wakatime_lab1}
\end{figure}

\subsection{对课程和实验的建议}

建议对实验设置一个及格分,达到分数以后实验才有效;建议每隔一段时间公布前 $10\%,~40\%$ 的实验效果,为同学们付出时间提供参考。

\end{document}