\documentclass{ctexart}
\usepackage{graphicx} % Required for inserting images
\usepackage{amsmath}
\usepackage{amssymb}
\usepackage{amsthm}
\usepackage{geometry}
\geometry{a4paper, left=1in, right=1in, top=1in, bottom=1in}

\title{《人工智能与机器学习基础》第一次实验报告}
\author{PB24000150 李欣宸}
\date{\today}

\begin{document}

\maketitle

\section{实验经过}

\subsection{实验过程}

\begin{itemize}
    \item 填写代码的空白,补全了 \verb|submission.py| 中线性回归和逻辑回归的代码;
    \item 测试线性回归时发现梯度爆炸,添加了 Weight Clipping 后训练成功,基本达到了解析解的水平(虽然训练步数太长,后续考虑进行规范化);
    \item 考虑到线性回归中 Weight Clipping 属于比较 Ad hoc 的策略,我对输入数据进行规范化处理,并移除了 Weight Clipping 的代码;
    \item 添加了 L1 和 L2 正则化的代码;
    \item 增加了特征数量,构造了对数,倒数特征;
    \item 发现之前写的解析解代码无法处理奇异矩阵,改用 \verb|np.linalg.pinv| 计算伪逆;
    \item 增加了复合特征,即原来的特征两两相乘的结果作为新的特征;
    \item 发现有些标准差为 0 的特征,规范化时会导致除 0 错误,添加了对标准差过小的特征不进行规范化的处理;
    \item 测试了逻辑回归的效果。
\end{itemize}

\subsection{超参数的调整}

\subsection{正则化处理}

\subsection{其他改进方法}

\section{实验结果}

\subsection{思考与分析}

\section{课程反馈}

\subsection{时间消耗}

% \subsection{对课程/实验/作业的建议}

\end{document}