\documentclass{ctexart}
\usepackage{graphicx} % Required for inserting images
\usepackage{amsmath}
\usepackage{amssymb}
\usepackage{amsthm}
\usepackage{geometry}
\geometry{a4paper, left=1in, right=1in, top=1in, bottom=1in}

\title{《人工智能与机器学习基础》第一次作业}
\author{PB24000150 李欣宸}
\date{\today}

\begin{document}
\maketitle

\section{第一题}

\subsection{(a)}

\newcommand{\et}{\mathbb{E}_{\mathrm{train}}}
\newcommand{\eyx}{\mathbb{E}_{y|\boldsymbol{x}}}
\newcommand{\fwt}{f_{\hat{w}(\mathrm{train})}}
\newcommand{\ft}{f_{\mathrm{true}}}
\newcommand{\fb}{\overline{f}}
\newcommand{\x}{\boldsymbol{x}}

\begin{proof}

考虑对一个固定的 $\x$,推导过程如下:

\begin{equation}
    \begin{aligned}
        & \et[\eyx[ (y - \fwt(\x))^2 ]] \\
        &= \et[\eyx[ ((y - \ft) + (\ft - \fwt))^2 ]] \\
        &= \underbrace{\eyx[ \epsilon^2 ]}_{\sigma^2~\text{的定义}} + \underbrace{2\et[\eyx[ \epsilon(\ft(\x) - \fwt(\x)) ]]}_{\epsilon~\text{和}~\ft(\x) - \fwt(\x)~\text{相互独立}} + \et[\eyx[ (\ft(\x) - \fwt(\x))^2 ]] \\
        &= \sigma^2 + 2\underbrace{\eyx[\epsilon]}_{=0}\et[ (\ft(\x) - \fwt(\x))] + \et[ (\ft(\x) - \fwt(\x))^2 ] \\
        &= \sigma^2 + \et[ (\ft(\x) - \fwt(\x))^2 ]
    \end{aligned}
    \label{1a1}
\end{equation}

而对于 $\et[ (\ft(\x) - \fwt(\x))^2 ]$,我们又有:

\begin{equation}
    \begin{aligned}
        & \et[ (\ft(\x) - \fwt(\x))^2 \\
        &= \et[ ((\ft(\x) - \fb(\x)) + (\fb(\x) - \fwt(\x)))^2 ] \\
        &= \underbrace{\et[ (\ft(\x) - \fb(\x))^2 ]}_{\mathrm{Bias}^2~\text{的定义}} + 2(\ft(\x) -\fb(\x))\underbrace{\et[ (\fb(\x) - \fwt(\x)) ]}_{\text{由}~\fb~\text{的定义,}=0} \\ &\quad+ \underbrace{\et[ (\fb(\x) - \fwt(\x))^2 ]}_{\mathrm{Var} 的定义} \\
        &= \mathrm{Bias}^2 + \mathrm{Var}
    \end{aligned}
    \label{1a2}
\end{equation}

综合 \eqref{1a1} 和 \eqref{1a2},我们得到:

\begin{equation}
    \et[\eyx[ (y - \fwt(\x))^2 ]] = \sigma^2 + \mathrm{Bias}^2 + \mathrm{Var} \qedhere
\end{equation}

\end{proof}

\subsection{(b)}

由 $\mathrm{Bias}^2$ 的定义 $\mathrm{Bias}^2 = \et[(\ft(\x) - \fb(\x))^2]$,由于平方项的存在,$\mathrm{Bias}^2 \ge 0$ 恒成立。

当 $\ft \ne \fb$ 时,即模型的表示力不足以表达数据产生的规律时,$\mathrm{Bias}^2$ 永远大于 0,一个直观的例子就是,一条直线不能拟合不在同一直线上的 3 个点。

\subsection{(c)}

\begin{itemize}
    \item $\mathrm{Bias}^2$ 只与我们建立的数学模型能否表示数据真实产生的机制有关,与 $N$ 和 $n$ 并无直接关联;但是若采取题目中 $\fb$ 的定义,$\mathrm{Bias}^2$ 与 $n$ 的关系是:随着 $n$ 的增大逐渐向下趋于一个定值,这个定值与模型的表示力和 $N$ 的大小有关;
    \item $\mathrm{Var} = \mathcal O(\frac{1}{N})$
\end{itemize}

模型越复杂,$\mathrm{Bias}^2$ 越低,$\mathrm{Var}$ 越高;反之亦然。

\section{第二题}

\subsection{(a)}

\newcommand{\err}{\operatorname{err}}

\begin{proof}

根据 Hoeffding 不等式,对于独立有界随机变量 \(Z_i \in [a, b]\),有:

\begin{equation}
    \Pr \left[ \left| \frac{1}{n} \sum_{i=1}^n Z_i - \mathbb{E}[Z_i] \right| > \epsilon \right] \leq 2 \exp \left( -\frac{2n\epsilon^2}{(b-a)^2} \right)
\label{hoeffding}
\end{equation}

取 $Z_i = I(h(x_i) \ne y_i)$,由定义,它是一个有界随机变量,$Z_i\in [0, 1]$。将 $a=0,b=1$ 带入 \eqref{hoeffding} 得:

\begin{equation}
    \Pr \left[ \left| \frac{1}{n} \sum_{i=1}^n Z_i - \mathbb{E}[Z_i] \right| > \epsilon \right] \leq 2 \exp (-2n\epsilon^2)
\end{equation}

而由 $Z_i$ 定义,经验误差 \(\err_S(h) = \frac{1}{n} \sum_{i=1}^n Z_i\),期望误差 \(\err_D(h) = \mathbb{E}[Z_i]\),则:

\begin{equation}
    \Pr[(\err_S(h) - \err_D(h)) > \epsilon] \le 2\exp(-2n\epsilon^2) \qedhere
\end{equation}

\end{proof}

\subsection{(b)}

\begin{proof}

\begin{equation}
    \begin{aligned}
        \Pr[\exists h \in H, (\err_S(h) - \err_D(h)) > \epsilon] &\le \sum_{h\in H} \Pr[(\err_S(h) - \err_D(h)) > \epsilon] \\
        &\le |H| \cdot 2 \exp(-2n\epsilon^2) \\
        &= 2|H|\exp(-2n\epsilon^2)
    \end{aligned}
    \label{22b}
    \qedhere
\end{equation}

\end{proof}

\subsection{(c)}

我们的数据偏差与训练集大小 $n$,模型类的大小 $|H|$ 有关:

\begin{itemize}
    \item 训练集大小 $n$:由 \eqref{22b},其他条件不变时,$h$ 的泛化性不好的概率随 $n$ 的增长指数下降。这说明 $n$ 的大小对模型的泛化性非常显著;
    \item 模型类的大小 $|H|$:由于 $|H|$ 的大小随模型参数量的增加指数增长;由 \eqref{22b},$h$ 泛化性不好的概率随模型类的大小线性增长,随模型参数量的增长指数增长。
\end{itemize}

这说明,训练参数量越大的模型,需要数据集大小也越多;如果认为 \eqref{22b} 给出的上界是一个紧的界,那么需要的数据集大小和参数量之间大概呈现出一个正比例关系。

为了减缓数据偏差带来的影响,我们要根据拥有数据集的大小选择合适的模型大小和正则化参数。

\end{document}