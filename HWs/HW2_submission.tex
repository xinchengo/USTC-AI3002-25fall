\documentclass{ctexart}
\usepackage{graphicx} % Required for inserting images
\usepackage{amsmath}
\usepackage{amssymb}
\usepackage{siunitx}
\usepackage{amsthm}
\usepackage{booktabs}
\usepackage{mathtools}
\usepackage{hyperref}
\usepackage{geometry}
\geometry{a4paper, left=1in, right=1in, top=1in, bottom=1in}

\allowdisplaybreaks

\title{《人工智能与机器学习基础》第二次作业}
\author{PB24000150 李欣宸}
\date{\today}

\begin{document}

\maketitle

\section{第一题}

\subsection{(a)}

记我们的原始数据矩阵为 $X_{\textrm{raw}}$,我们有归一化公式:
\begin{equation}
    X = D^{-1} \widetilde{X}
\end{equation}

其中($J_{mm} \in \mathbb{R}^{m \times m}$ 是全 1 矩阵):
\begin{equation}
    \widetilde{X} = X_{\textrm{raw}} - \frac{1}{m} X_{\textrm{raw}} J_{mm}, \quad D = \mathrm{diag}\left( \sqrt{\frac{1}{m}( \widetilde{X} \odot \widetilde{X} ) \mathbf{1}_m} \right)
\end{equation}

在很多数据中,特征的绝对大小没有实际意义。“中心化”起到的作用就是消除不同特征的量纲和绝对大小的影响,使得 PCA 的结果更能反映实际。

\subsection{(b)}

首先我们知道 $x$ 在方向 $u$ 上的投影为 $(x^T u) u$,我们想在 $u^Tu=1$ 的条件下最小化投影点和原始点的均方误差,使用 Lagrange 乘子法:
\begin{gather}
\begin{aligned}
    \mathcal{L}(u, \lambda) &= \sum_{i=1}^m \left\|x^{(i)} - (x^{(i)T} u) u\right\|^2 + \lambda (u^T u - 1) \\
    &= \sum_{i=1}^m \left( x^{(i)T} x^{(i)} - 2 (x^{(i)T} u)^2 + (x^{(i)T} u)^2 (u^T u) \right) + \lambda (u^T u - 1) \\
    &= \sum_{i=1}^m x^{(i)T} x^{(i)} - \sum_{i=1}^m (x^{(i)T} u)^2 + \lambda (u^T u - 1)
\end{aligned} \\
\frac{\partial \mathcal{L}}{\partial u} = -2 \sum_{i=1}^m (x^{(i)T} u) x^{(i)} + 2 \lambda u = -2 X^T X u + 2 m \lambda u = 0 \\
\implies \quad \boxed{\frac{1}{m} X^T X u = \lambda u} \label{eq:eigen}
\end{gather}

把这一结果带回到 $\mathcal{L}$ 中:
\begin{gather}
    \eqref{eq:eigen} \implies u^T X^T X u = m \lambda u^T u = m \lambda \\
    \implies \mathcal{L}(u, \lambda) = \sum_{i=1}^m x^{(i)T} x^{(i)} - m \lambda + \lambda (u^T u - 1) = \sum_{i=1}^m x^{(i)T} x^{(i)} - m \lambda \label{eq:lagrange}
\end{gather}


由于 $X$ 是经过归一化的数据,所以 $S = \frac{1}{m} X^T X$ 就是数据的协方差矩阵;因此可能的 $\lambda$ 和 $u$ 就是协方差矩阵的特征值和特征向量。为了最小化 \eqref{eq:lagrange},我们需要选用最大的特征值对应的特征向量作为 $u$——即,最小化均方误差的单位长度向量 $u$ 是对应于数据的第一个主成分。

\section{第二题第一部分}

\subsection{(a)}

\begin{table}[htbp]
    \centering
    \caption{第一个 $k$-means 算法的迭代过程}
    \label{k-means-1}
    \begin{tabular}{ccccc}
        \toprule
        迭代次数 & 簇 1 中的下标 & 簇 2 中的下标 & 新的簇 1 中心 & 新的簇 2 中心 \\
        \midrule
        1 & $\{1, 2, 3, 4\}$ & $\{5, 6, 7, 8\}$ & $(1.5, 1.5)$ & $(10.5, 10.5)$ \\
        \bottomrule
    \end{tabular}
\end{table}

算法在第一次迭代后收敛,最终的类中心为 $(1.5, 1.5)$ 和 $(10.5, 10.5)$。
\subsection{(b)}

\begin{table}[htbp]
    \centering
    \caption{第二个 $k$-means 算法的迭代过程}
    \label{k-means-2}
    \begin{tabular}{ccccc}
        \toprule
        迭代次数 & 簇 1 中的下标 & 簇 2 中的下标 & 新的簇 1 中心 & 新的簇 2 中心 \\
        \midrule
        1 & $\{1, 2, 3\}$ & $\{4, 5, 6, 7, 8\}$ & $\left(\frac{4}{3}, \frac{4}{3}\right)$ & $(8.8, 8.8)$ \\
        2 & $\{1, 2, 3, 4\}$ & $\{5, 6, 7, 8\}$ & $(1.5, 1.5)$ & $(10.5, 10.5)$ \\
        \bottomrule
    \end{tabular}
\end{table}

算法在第二次迭代后收敛\footnote{这题的运算中会遇到数据点离两个簇中心的距离相等的情况,可能导致簇的划分不唯一;在这里统一认为数据点会被划分到编号较小的簇中},最终的类中心为 $(1.5, 1.5)$ 和 $(10.5, 10.5)$。

\subsection{(c)}

解决初始中心选择问题的方法有但不限于这些:

\begin{enumerate}
    \item 多次运行 $k$-means 算法,选择样本到簇中心距离平方和最小的一次结果作为最终结果;
    \item 使用 $k$-means++ 算法选择初始中心,增加初始中心之间的散布程度,减少陷入局部最优的可能性;
    \item 对数据进行预处理,去除明显的异常值;
    \item 使用层次聚类(hierarchical clustering)算法,逐步增加簇的数量,用上一步算法的结果加以扰动作为下一步的初始中心。
\end{enumerate}

\subsection{(d)}

每个点被选中的概率与其与 $c_1$ 的距离平方成正比,结果如表 \ref{k-means-pp}:

\begin{table}[htbp]
    \centering
    \caption{$k$-means++算法中每个点被选中为第二个类中心的概率}
    \label{k-means-pp}
    \begin{tabular}{ccccccccc}
        点 & (1,1) & (1,2) & (2,1) & (2,2) & (10, 10) & (10, 11) & (11, 10) & (11, 11) \\
        \midrule
        距离平方 & 0 & 1 & 1 & 2 & 162 & 181 & 181 & 200 \\
        概率(\%) & \num{0} & \num{0.14} & \num{0.14} & \num{0.27} & \num{22.25} & \num{24.86} & \num{24.86} & \num{27.47} \\
        概率(分数) & $\tfrac{0}{1}$ & $\tfrac{1}{728}$ & $\tfrac{1}{728}$ & $\tfrac{1}{364}$ & $\tfrac{81}{364}$ & $\tfrac{181}{728}$ & $\tfrac{181}{728}$ & $\tfrac{25}{91}$ \\
        \bottomrule
    \end{tabular}
\end{table}
\subsection{(e)}

每个点被选中的概率与其与 $c_1$、$c_2$ 距离平方的最小值成正比,结果如表 \ref{k-means-pp-2}:

\begin{table}[htbp]
    \centering
    \caption{$k$-means++算法中每个点被选中为第三个类中心的概率}
    \label{k-means-pp-2}
    \begin{tabular}{ccccccccc}
        点 & (1,1) & (1,2) & (2,1) & (2,2) & (10, 10) & (10, 11) & (11, 10) & (11, 11) \\
        \midrule
        距离平方最小值 & 0 & 1 & 1 & 2 & 0 & 1 & 1 & 2 \\
        概率(\%) & \num{0} & \num{12.5} & \num{12.5} & \num{25} & \num{0} & \num{12.5} & \num{12.5} & \num{25} \\
        概率(分数) & $\tfrac{0}{1}$ & $\tfrac{1}{8}$ & $\tfrac{1}{8}$ & $\tfrac{1}{4}$ & $\tfrac{0}{1}$ & $\tfrac{1}{8}$ & $\tfrac{1}{8}$ & $\tfrac{1}{4}$ \\
        \bottomrule
    \end{tabular}
\end{table}

\section{第二题第二部分}
\subsection{(a)}

虽然对于核 $k$-均值聚类我们往往无法显式计算 $\mu_i$,但是我们仍然有特征空间中簇中心的求解公式:
\begin{equation}
    \mu_{i} = \frac{1}{|C_i|}\sum_{x\in C_i} \phi(x)
\end{equation}

\subsection{(b)}

\begin{gather}
\left\| \phi(x) - \mu_i \right\|^2 = K(x, x) - \frac{2}{|C_i|}\sum_{y\in C_i}K(x, y) + \frac{1}{|C_i|^2}\sum_{y\in C_i}\sum_{z\in C_i} K(y, z) \\
\begin{aligned}
    \implies \quad c &= \arg \min_{i}  \left\| \phi(x) - \mu_i \right\|^2 \\
    &= \arg \min_{i} \left[ \frac{1}{|C_i|^2}\sum_{y\in C_i}\sum_{z\in C_i} K(y, z) - \frac{2}{|C_i|}\sum_{y\in C_i}K(x, y)\right]
\end{aligned}
\end{gather}

这个算法中没有出现任何单独的 $\phi(x)$ 和 $\mu_i$ 项,只需要预先计算所有点对之间的核函数。

\section{第三题}
\subsection{(a)}

隐 Markov 模型中的 EM 算法称为 Baum-Welch 算法,下面给出 Baum-Welch 算法的简单推导过程,先定义一些符号:
\begin{itemize}
    \item $X = (x_1, x_2, \ldots, x_T)$ 观察序列;
    \item $Z = (z_1, z_2, \ldots, z_T)$ 对应的隐状态序列;
    \item $\Phi=\{\phi_j\}_{j=1}^K$ 为发射参数,其中 $\phi_j$ 对 $p(x_t|z_t-j,\phi_j)$ 参数化; 
    \item $A = \{a_{ij}\}_{i,j=1}^K$ 为转移概率;
    \item $\pi = \{\pi_i\}_{i=1}^K$ 为初始状态分布。
\end{itemize}

由一般 EM 算法中 Q 函数的定义:
\begin{equation}
    Q(\theta|\theta^{(n)}) = \mathbb{E}_{Z|X,\theta^{(n)}} [\log P(X, Z|\theta)] \label{q-func}
\end{equation}

由 Markov 性质,后验分布下的对数似然可以写成:
\begin{gather}
    \log P(X, Z|\theta) = \log \pi_{z_1} + \sum_{t=1}^{T-1} \log A_{z_t, z_{t+1}} + \sum_{t=1}^{T} \log p(x_t|z_t, \phi_{z_t}) \\
    \begin{aligned}
    \implies Q(\theta|\theta^{(n)}) &= \sum_{z_1} P(z_1|X,\theta^{(n)}) \log \pi_{z_1} + \sum_{t=1}^{T-1} \sum_{i=1}^{K} \sum_{j=1}^{K} P(z_t=i, z_{t+1}=j|X,\theta^{(n)}) \log a_{ij} \\ 
    &\quad + \sum_{t=1}^{T} \sum_{i=1}^{K} P(z_t=i|X,\theta^{(n)}) \log p(x_t|z_t=i, \phi_i) \\
    &\triangleq Q_{\pi} + Q_{A} + Q_{\Phi}
    \end{aligned}\\
    \gamma_t(i) \triangleq P(z_t=i|X,\theta^{(n)}), \quad \xi_t(i,j) \triangleq P(z_t=i, z_{t+1}=j|X,\theta^{(n)}) \label{gamma-xi}\\
    \begin{aligned}
    \implies Q(\theta|\theta^{(n)}) &= \sum_{i=1}^{K} \gamma_1(i) \log \pi_i + \sum_{t=1}^{T-1} \sum_{i=1}^{K} \sum_{j=1}^{K} \xi_t(i,j) \log a_{ij} + \sum_{t=1}^{T} \sum_{i=1}^{K} \gamma_t(i) \log p(x_t|z_t=i, \phi_i)
    \end{aligned} \label{q-decomposed}
\end{gather}

由 \eqref{q-decomposed} 可以看出,\eqref{gamma-xi} 中的 $\gamma_t(i)$ 和 $\xi_t(i,j)$ 可以确定 Q 函数。

由一般的 EM 算法定义,E 步中我们要根据当前参数计算 $\gamma_t(i)$ 和 $\xi_t(i,j)$,从而计算 Q 函数;M 步中我们要最大化 Q 函数以更新参数估计 $\theta^{(n+1)}$。

\subsubsection{Baum-Welch 算法的 E 步}
我们考虑如何计算 $\gamma_t(i)$ 和 $\xi_t(i,j)$。由于 X 是序列信息,我们考虑是否可以用递推的方法计算某种前缀和后缀信息,并把两种信息进行合并;这就是前向-后向算法的基本思想,定义:
\begin{align}
    \alpha_t(i) &\triangleq p(X[1:t], z_t=i|\theta^{(n)}) \\
    \beta_t(i) &\triangleq p(X[t+1:T]|z_t=i,\theta^{(n)})
\end{align}

则边缘化后验可以写成:
\begin{gather}
    \gamma_t(i) = \frac{p(X,z_t=i|\theta^{(n)})}{p(X|\theta^{(n)})} = \frac{\alpha_t(i) \beta_t(i)}{p(X|\theta^{(n)})} \label{gamma}\\
    \xi_t(i,j) = \frac{p(X,z_t=i,z_{t+1}=j|\theta^{(n)})}{p(X|\theta^{(n)})} = \frac{\alpha_t(i) a_{ij} p(x_{t+1}|z_{t+1}=j, \phi_j) \beta_{t+1}(j)}{p(X|\theta^{(n)})} \label{xi}
\end{gather}

而其中的 $\alpha_t(i)$ 和 $\beta_t(i)$ 可以按照以下递推式计算得到:
\begin{align}
    \alpha_1(i) &= \pi_i p(x_1|z_1=i, \phi_i) \notag \\
    \alpha_{t+1}(j) &= \left[ \sum_{i=1}^{K} \alpha_t(i) a_{ij} \right] p(x_{t+1}|z_{t+1}=j, \phi_j) \label{alpha-recursion}\\
    \beta_T(i) &= 1 \notag \\
    \beta_t(i) &= \sum_{j=1}^{K} a_{ij} p(x_{t+1}|z_{t+1}=j, \phi_j) \beta_{t+1}(j) \label{beta-recursion}
\end{align}

\subsubsection{Baum-Welch 算法的 M 步}

\begin{itemize}
    \item 对于 $\pi$,我们要最大化 $Q_{\pi} = \sum_{i=1}^{K} \gamma_1(i) \log \pi_i$,加上约束条件 $\sum_{i=1}^{K} \pi_i = 1,~\pi_i \ge 0$,使用 Lagrange 乘子法:
    \begin{gather}
        \mathcal{L}(\pi, \lambda) = \sum_{i=1}^{K} \gamma_1(i) \log \pi_i + \lambda \left( \sum_{i=1}^{K} \pi_i - 1 \right) \\
        \frac{\partial \mathcal{L}}{\partial \pi_i} = \frac{\gamma_1(i)}{\pi_i} + \lambda = 0 \implies \pi_i = -\frac{\gamma_1(i)}{\lambda} \\
        \implies \sum_{i=1}^{K} \pi_i = -\frac{1}{\lambda} \sum_{i=1}^{K} \gamma_1(i) = 1 \implies \lambda = -\sum_{i=1}^{K} \gamma_1(i) = -1 \\
        \implies \boxed{\pi_i^{(n+1)} = \gamma_1(i)} \label{pi-update}
    \end{gather}
    \item 对于 $A$,我们要最大化 $Q_{A} = \sum_{t=1}^{T-1} \sum_{i,j} \xi_t(i,j) \log a_{ij}$,加上约束条件 $\sum_{j=1}^{K} a_{ij} = 1,~a_{ij} \ge 0$,使用 Lagrange 乘子法:
    \begin{gather}
        \mathcal{L}(A, \{\lambda_i\}) = \sum_{t=1}^{T-1} \sum_{i=1}^{K} \sum_{j=1}^{K} \xi_t(i,j) \log a_{ij} + \sum_{i=1}^{K} \lambda_i \left( \sum_{j=1}^{K} a_{ij} - 1 \right) \\
        \frac{\partial \mathcal{L}}{\partial a_{ij}} = \sum_{t=1}^{T-1} \frac{\xi_t(i,j)}{a_{ij}} + \lambda_i = 0 \implies a_{ij} = -\frac{\sum_{t=1}^{T-1} \xi_t(i,j)}{\lambda_i} \\
        \implies \sum_{j=1}^{K} a_{ij} = -\frac{1}{\lambda_i} \sum_{j=1}^{K} \sum_{t=1}^{T-1} \xi_t(i,j) = 1 \implies \lambda_i = -\sum_{j=1}^{K} \sum_{t=1}^{T-1} \xi_t(i,j) = -\sum_{t=1}^{T-1} \gamma_t(i) \\
        \implies \boxed{a_{ij}^{(n+1)} = \frac{\sum_{t=1}^{T-1} \xi_t(i,j)}{\sum_{t=1}^{T-1} \gamma_t(i)}} \label{a-update}
    \end{gather}
    \item 对于 $\Phi$,我们要最大化 $Q_{\Phi} = \sum_{t=1}^{T} \sum_{i=1}^{K} \gamma_t(i) \log p(x_t|z_t=i, \phi_i)$
    
    该题中的发射形式为离散发射,观察值可能的取值为字母表 $\mathcal{V} = \{v_1, v_2, \ldots, v_M\}$,观察分布 $B_j(x) = p(x|z_t=j, \phi_j)=b_j(x)$ 满足 $\sum_{k=1}^{M} b_j(v_k) = 1,~b_j(v_k) \ge 0$。使用 Lagrange 乘子法:
    \begin{gather}
        \mathcal{L}(\Phi, \{\lambda_j\}) = \sum_{t=1}^{T} \sum_{i=1}^{K} \gamma_t(i) \log b_i(x_t) + \sum_{j=1}^{K} \lambda_j \left( \sum_{k=1}^{M} b_j(v_k) - 1 \right) \\
        \frac{\partial \mathcal{L}}{\partial b_j(v_k)} = \sum_{t:x_t=v_k} \frac{\gamma_t(j)}{b_j(v_k)} + \lambda_j = 0 \implies b_j(v_k) = -\frac{\sum_{t:x_t=v_k} \gamma_t(j)}{\lambda_j} \\
        \implies \sum_{k=1}^{M} b_j(v_k) = -\frac{1}{\lambda_j} \sum_{k=1}^{M} \sum_{t:x_t=v_k} \gamma_t(j) = -\frac{1}{\lambda_j} \sum_{t=1}^{T} \gamma_t(j) = 1 \implies \lambda_j = -\sum_{t=1}^{T} \gamma_t(j) \\
        \implies \boxed{b_j^{(n+1)}(v_k) = \frac{\sum_{t:x_t=v_k} \gamma_t(j)}{\sum_{t=1}^{T} \gamma_t(j)}} \label{b-update}
    \end{gather}
\end{itemize}

\subsubsection{本题的计算}

将 Sunny, Rainy 分别编号为 0,1;高温、低温分别编号为 0,1。观察序列 $X = (0, 1, 0, 1)$。则按照递推公式 \eqref{alpha-recursion} 和 \eqref{beta-recursion} 可以得到:
\begin{gather}
    (\alpha_t(i))_{i,t} = \begin{pmatrix}
    0.32  & 0.0492 & 0.066288 & 0.01054908 \\
    0.18 & 0.1778 & 0.043242  & 0.03974922
    \end{pmatrix} \\
    (\beta_t(i))_{i,t} = \begin{pmatrix}
        0.09018 & 0.258 & 0.4 & 1 \\
        0.119115 & 0.2115 & 0.55 & 1
    \end{pmatrix}
\end{gather}

根据 \eqref{gamma} 和 \eqref{xi},可以得到:
\begin{gather}
    (\gamma_t(i))_{i,t} = \begin{pmatrix}
    0.57372913 & 0.25236638 & 0.52715897 & 0.20973035 \\
    0.42627087 & 0.74763362 & 0.47284103 & 0.79026965
    \end{pmatrix}
\\
    (\xi_t(i, j)) = \left\{ \begin{pmatrix}
    0.19696888 & 0.37676025 \\
    0.0553975 & 0.37087337
    \end{pmatrix},
    \begin{pmatrix}
       0.18780754 & 0.06455884 \\
        0.33935143 & 0.40828219
    \end{pmatrix},
    \begin{pmatrix}
       0.15814769 & 0.36901128 \\
        0.05158266 & 0.42125837
    \end{pmatrix}
        \right\}
\end{gather}

根据 \eqref{pi-update},\eqref{a-update},\eqref{b-update},可以得到更新后的参数:
\begin{gather}
    \pi^{(1)} = \begin{pmatrix}
        0.57372913 & 0.42627087
    \end{pmatrix} \\
    A^{(1)} = \begin{pmatrix}
        0.40119883 & 0.59880117 \\
        0.27103859 & 0.72896141
    \end{pmatrix}
\\
    B^{(1)} = \begin{pmatrix}
        0.70434983 & 0.29565017 \\
        0.3689398 & 0.6310602
    \end{pmatrix}
\end{gather}

\subsection{(b)}

我们的目标是证明对数似然函数 $l(\theta) = \log p(O|\theta)$ 单调递增且有上界,从而证明 EM 算法的收敛性。

在 EM 算法中,对数似然,证据下界,Q 函数之间满足以下的关系:
\begin{gather}
    \log p(O|\theta) = \mathcal{L}(q, \theta) + \mathrm{KL}(q(Q) || p(Q|O,\theta)) \label{eq:elbo-decomp}\\
    \mathcal{L}(q, \theta) = Q(q, \theta) + H(q)
\end{gather}

其中 $\mathcal{L}(q, \theta)$ 为证据下界(ELBO),$Q(q, \theta) = \mathbb{E}_{Q \sim q}[\log p(O, Q|\theta)]$ 为 Q 函数,$H(q) = -\mathbb{E}_{Q \sim q}[\log q(Q)]$ 为熵。

\textbf{E 步:}由于我们选择了 $q^{(n)}(Q) = p(Q|O,\theta^{(n)})$,所以有:
\begin{gather}
    \mathrm{KL}(q^{(n)}(Q) || p(Q|O,\theta^{(n)})) = 0 \\
    \implies \quad \log p(O|\theta^{(n)}) = \mathcal{L}(q^{(n)}, \theta^{(n)})
\end{gather}

\textbf{M 步:}由于我们选择了 $\theta^{(n+1)} = \arg \max_{\theta} Q(\theta|\theta^{(n)}) = \arg \max_{\theta} Q(q^{(n)}, \theta)$(注意 $H(q^{(n)})$ 与 $\theta$ 无关),所以有:
\begin{gather}
    Q(q^{(n)}, \theta^{(n+1)}) \ge Q(q^{(n)}, \theta^{(n)}) \\
    \begin{aligned}
    \implies \mathcal{L}(q^{(n)}, \theta^{(n+1)}) &= Q(q^{(n)}, \theta^{(n+1)}) + H(q^{(n)}) \\
    &\ge Q(q^{(n)}, \theta^{(n)}) + H(q^{(n)}) \\
    &= \mathcal{L}(q^{(n)}, \theta^{(n)}) \\
    &= \log p(O|\theta^{(n)})
    \end{aligned}
\end{gather}

由 \eqref{eq:elbo-decomp} 和 KL 散度的非负性,我们有:
\begin{gather}
    \log p(O|\theta^{(n+1)}) = \mathcal{L}(q^{(n)}, \theta^{(n+1)}) + \mathrm{KL}(q^{(n)}(Q) || p(Q|O,\theta^{(n+1)})) \ge \mathcal{L}(q^{(n)}, \theta^{(n+1)}) \\
    \implies \log p(O|\theta^{(n+1)}) \ge \mathcal{L}(q^{(n)}, \theta^{(n+1)}) \ge \log p(O|\theta^{(n)})
\end{gather}

这表明对数似然函数 $l(\theta) = \log p(O|\theta)$ 在每次迭代后单调不减。又因为 $\log p(O|\theta) \le 0$(概率不超过 1),对数似然函数有上界。由于单调有界的数列必有极限,EM 算法必然收敛。

\subsection{(c)}

因为对数似然函数 $l(\theta) = \log p(O|\theta)$ 通常是一个复杂的非凸函数,EM 算法等价于证据下界的坐标下降法,所以不能保证收敛到全局最优解,只能保证收敛到某个局部最优解或鞍点。

解决方法有但不限于:
\begin{itemize}
    \item 多次运行 EM 算法,选择对数似然值最高的一次结果作为最终结果;
    \item 使用随机 EM 算法,在 E 步中使用采样的方法近似计算后验分布,从而增加算法的随机性,减少陷入局部最优的可能性。
\end{itemize}

\end{document}